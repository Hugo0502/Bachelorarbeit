\section{Usability-Tests}
\label{sec:usabilitytests}

% Tool / Websiten / Unternehmen die Usability Test anbieten

% auf die unterschiedlichen Arten von Tests eingehen -> automatisiert / Expertentests / Usertest
\subsection{Automatisierte Tools zur Bewertung der Usability von Websites - SEOptimiser}
\label{sec:automatisierte_tools_zur_bewertung_der_usability_von_websites_seooptimiser}

SEOptimiser ist ein automatisiertes Tool, das zur Bewertung der Usability von Websites eingesetzt wird und in der Studie "A Review of Automated Website Usability Evaluation Tools: Research Issues and Challenges" von Namaoun et al. (2021) analysiert wurde. Das Tool wurde entwickelt, um Webdesigner und Website-Betreiber bei der Untersuchung der Usability, User Experience (UX) und Suchmaschinenoptimierung (SEO) ihrer Websites zu unterstützen. SEOptimiser geht über einfache Diagnosen hinaus und liefert spezifische, umsetzbare Empfehlungen zur Verbesserung der Website-Qualität.

Das primäre Ziel von SEOptimiser ist die Identifizierung von Usability- und Performance-Problemen sowie die Bereitstellung spezifischer Empfehlungen zur Verbesserung. Das Tool deckt dabei verschiedene Schlüsselbereiche ab, darunter SEO, Benutzerfreundlichkeit, Leistung, soziale Interaktion und Sicherheit. Diese umfassende Analyse macht SEOptimiser zu einer vielseitigen Lösung für die Webanalyse. Es ist bemerkenswert, dass SEOptimiser sechs verschiedene Dimensionen der Benutzerfreundlichkeit abdeckt, mehr als viele vergleichbare Tools. Zu den analysierten Bereichen gehören der Seiteninhalt, Ladegeschwindigkeit, Effizienz, Barrierefreiheit, Interaktivität und Navigation. Diese Dimensionen sind zentral für die User Experience, da sie die Art und Weise beeinflussen, wie Nutzer mit der Website interagieren. Zusätzlich bietet SEOptimiser detaillierte Einblicke in die SEO-Leistung, einschließlich der Seitenoptimierung, der Nutzung von Schlüsselwörtern und Meta-Beschreibungen.

SEOptimiser legt einen besonderen Fokus auf die Verbesserung der Usability und der User Experience. Die detaillierten Analysen helfen, Schwachstellen in der Benutzerführung und im Interface-Design zu identifizieren und konkrete Verbesserungsvorschläge zu machen. Zu den analysierten Dimensionen gehören Aspekte wie Navigation, Effizienz der Benutzerführung und Interaktivität der Website. Diese Faktoren sind entscheidend für eine gute UX, da sie sicherstellen, dass Nutzer die benötigten Informationen schnell finden, sich problemlos orientieren und insgesamt eine positive Interaktion mit der Website haben.

Die Empfehlungen von SEOptimiser zur Verbesserung der Usability konzentrieren sich auf konkrete Maßnahmen zur Optimierung der Benutzerinteraktion, zur Beseitigung technischer Probleme und zur Erhöhung der Zugänglichkeit. Im Gegensatz zu einigen anderen Tools, die nur allgemeine Bewertungen und vage Analysen bieten, liefert SEOptimiser detaillierte Diagnosen, die für Benutzer mit moderaten technischen Kenntnissen gut verständlich sind. Die Berichte des Tools sind übersichtlich gestaltet, was es auch für Nutzer ohne speziellen Hintergrund in der Website-Entwicklung einfacher macht, die Ergebnisse zu verstehen und anzuwenden.

Darüber hinaus unterstützt SEOptimiser die Verbesserung der User Experience, indem es mögliche Engpässe identifiziert, wie z. B. übergroße Seitenelemente, ineffiziente Programmierpraktiken oder mangelnde Barrierefreiheit, und konkrete Lösungen vorschlägt. Diese Art der Analyse ermöglicht es Website-Betreibern, die User Experience gezielt zu verbessern und sicherzustellen, dass alle Nutzer, unabhängig von ihren technischen Fähigkeiten oder Einschränkungen, eine möglichst positive Erfahrung machen.

Trotz seiner Stärken weist SEOptimiser auch Einschränkungen auf, insbesondere im Hinblick auf die theoretische Fundierung der Usability-Analysen. Ein Problem, das in der Studie festgestellt wurde, ist, dass SEOptimiser, ähnlich wie viele andere Tools, die theoretischen Grundlagen der Usability nicht vollständig in sein Analyse-Framework integriert. Das Tool bewertet eine begrenzte Anzahl von Usability-Dimensionen im Vergleich zu dem breiteren Spektrum, das in der akademischen Literatur anerkannt ist. Aspekte wie Fehlertoleranz, Erlernbarkeit und Benutzerzufriedenheit werden beispielsweise nicht berücksichtigt. Diese Lücken können dazu führen, dass die Usability einer Website nicht vollständig erfasst wird, insbesondere bei komplexen, nutzerzentrierten Designs, die eine tiefgehende Analyse des Nutzerverhaltens erfordern.

Ein weiteres Problem ist die Variabilität der Analyseergebnisse im Vergleich zu anderen Tools. Obwohl SEOptimiser in seinen Empfehlungen konsistent ist, gab es Fälle, in denen die Bewertungen stark von denen anderer bekannter Tools abwichen. Dies wirft Fragen zur Zuverlässigkeit und Standardisierung der Bewertungsmetriken auf. Diese Inkonsistenzen zeigen, dass eine engere Zusammenarbeit zwischen Branchenexperten und HCI-Forschern (Human-Computer Interaction) notwendig ist, um die Zuverlässigkeit und Validität automatisierter Usability-Tools wie SEOptimiser zu verbessern.

SEOptimiser ist ein vielseitiges Tool zur Bewertung der Usability und User Experience von Websites. Es zeichnet sich durch seinen umfassenden Analyseumfang, klare Berichte und umsetzbare Empfehlungen aus. Der Fokus auf SEO, Benutzerfreundlichkeit und die Integration sozialer Medien machen SEOptimiser zu einer wertvollen Ressource für die Verbesserung der Website-Qualität in verschiedenen Bereichen. Dennoch gibt es einige Herausforderungen, insbesondere bei der vollständigen Abdeckung aller Aspekte der Usability und der Sicherstellung der Konsistenz der Bewertungsmetriken. Die Studie schlägt vor, dass eine stärkere Verbindung zwischen den theoretischen Grundlagen der Usability und der praktischen Implementierung in Tools den Nutzen von Lösungen wie SEOptimiser im Bereich der automatisierten Usability-Tests weiter erhöhen könnte.

% auch den Baymard Score aufführen
\subsection{Baymard Institute}
\label{sec:baymard_institute}

\subsubsection{Baymard Institute: Überblick und Bedeutung}
\label{ueblick_und_bedeutung}

Das Baymard Institute (UX Audit – Baymard Institute, o. D.) ist eine Forschungseinrichtung, die sich auf User Experience (UX) spezialisiert hat und umfangreiche Erfahrung mit mehr als 150.000 Stunden Forschung gesammelt hat. Das Institut unterstützt Unternehmen dabei, ihre Webseiten zu analysieren und die Benutzerfreundlichkeit zu verbessern. Der Schwerpunkt liegt dabei auf der Identifikation von Schwachstellen im UX-Bereich sowie auf der Entwicklung praxisorientierter Empfehlungen zur Optimierung der Nutzererfahrung.

Ein wichtiger Bestandteil des Angebots ist der Audit-Service, der Unternehmen Möglichkeiten zur Verbesserung der UX aufzeigt. Im Rahmen eines Audits werden systematisch die 40 bedeutendsten UX-Probleme eines Unternehmens identifiziert. Diese Probleme können von größeren Schwächen bis hin zu kleineren Details der Benutzerfreundlichkeit reichen, die in ihrer Summe die Nutzererfahrung beeinträchtigen. Darüber hinaus bietet das Baymard Institute eine Möglichkeit, die UX-Leistung eines Unternehmens mit führenden Marken und Branchenstandards zu vergleichen. Diese Benchmarking-Analyse bietet eine fundierte Grundlage, um die eigene Performance im Vergleich zur Konkurrenz zu bewerten und darauf aufbauend strategische Entscheidungen zu treffen. Dadurch erhalten Unternehmen eine bessere Übersicht über ihre Marktposition und können gezielte Maßnahmen zur Steigerung der Nutzerfreundlichkeit umsetzen. Zusätzlich unterstützt der Audit-Prozess die Ressourcenplanung, indem er eine Roadmap zur Optimierung von UX- und Entwicklungsressourcen bereitstellt. Diese Roadmap fokussiert sich auf Bereiche, die den höchsten Return on Investment (ROI) versprechen. Die Dokumentation der UX-Leistung in Form jährlicher Berichte hilft zudem dabei, die Verbesserungen der Nutzererfahrung nachvollziehbar zu machen und die Ergebnisse gegenüber Stakeholdern transparent darzustellen.

Die Audits des Baymard Instituts sind auf die spezifischen Anforderungen der jeweiligen Branche zugeschnitten. Unternehmen haben die Möglichkeit, aus verschiedenen Kategorien auszuwählen, zum Beispiel "Digital Subscriptions \& SaaS", "Food Delivery \& Takeout" oder "Sports Gear \& Equipment". Diese Branchenspezialisierung ermöglicht eine gezielte Adressierung spezifischer Herausforderungen und die Entwicklung passender Lösungsansätze.

Ein typisches Audit des Baymard Instituts besteht aus verschiedenen Elementen. Zunächst erfolgt eine tiefgehende UX-Analyse, bei der erfahrene UX-Forscher eine umfassende Analyse der Desktop- und mobilen Webseiten durchführen. Diese Analyse basiert auf den umfangreichen Forschungsergebnissen, die das Institut in über 150.000 Stunden Forschung gesammelt hat. Auf Grundlage dieser Analyse wird ein detaillierter Bericht erstellt, der über 120 Seiten umfasst. Dieser Bericht enthält 40 forschungsbasierte Empfehlungen, die sowohl bestehende UX-Probleme als auch konkrete Lösungsansätze beschreiben. Zusätzlich werden Best Practice-Beispiele von führenden Webseiten vorgestellt, die als Orientierungshilfe dienen können. Ein weiteres Element des Audits sind die UX-Scorecards, die eine detaillierte Leistungsbewertung ermöglichen. Diese Scorecards enthalten über 500 Bewertungskriterien und dienen dazu, die UX-Leistung des Unternehmens mit der von 251 führenden US- und europäischen Webseiten zu vergleichen. Im Anschluss an das Audit wird eine  Videokonferenz angeboten, um die Ergebnisse und vorgeschlagenen Verbesserungsmaßnahmen im Detail mit dem Team des Unternehmens zu besprechen. Dieser Austausch trägt dazu bei, dass die Maßnahmen besser verstanden und gezielt umgesetzt werden können.

Nach Abschluss des Audits bietet das Baymard Institut weiterhin Unterstützung an. Es werden Follow-up-Gespräche angeboten, bei denen das Unternehmen Feedback zu Neugestaltungen, Prototypen oder weiteren Fragen einholen kann. Dieser kontinuierliche Austausch stellt sicher, dass die vorgeschlagenen Maßnahmen erfolgreich implementiert werden und die UX langfristig verbessert wird.

Zusätzlich zum Audit erhalten Mitarbeitende des Unternehmens Zugang zu Baymard Premium. Dieses Angebot beinhaltet Zugriff auf alle Ergebnisse der UX-Forschung, UX-Zertifizierungen und weiteres Material. Dadurch erhalten Unternehmen Zugang zu wertvollen Informationen, die sonst kostenpflichtig wären.

Das Baymard Institut bietet Unternehmen die Möglichkeit, die UX ihrer Webseiten umfassend zu analysieren und gezielt zu verbessern. Der Fokus auf praxisorientierte Forschung sowie die enge Zusammenarbeit mit den Unternehmen tragen dazu bei, die Nutzerfreundlichkeit systematisch und nachhaltig zu optimieren. Die branchenspezifischen Audits und die detaillierte Analyse helfen Unternehmen, gezielte Maßnahmen zu ergreifen und ihre Position im Markt zu stärken.


\subsubsection{UX-Audit-Prozess}
\label{ux_audit_prozess}

Der UX-Audit-Prozess des Baymard Instituts beinhaltet eine umfassende Analyse der Usability unterschiedlicher Bereiche einer Webseite. Der Schwerpunkt liegt auf der detaillierten Evaluierung der Nutzererfahrung, wobei jeder Schritt der Nutzerinteraktion präzise untersucht wird. Die Audits sind modular aufgebaut und decken die zentralen Elemente einer Webseite ab, um sicherzustellen, dass die Benutzeroberfläche benutzerfreundlich, zugänglich und intuitiv gestaltet ist. Das Ziel des Prozesses ist es, Schwachstellen zu identifizieren und konkrete Verbesserungsvorschläge zu entwickeln, die zur kontinuierlichen Optimierung der Nutzerfreundlichkeit beitragen. Die Analyse basiert auf Forschungsergebnissen und praxisnahen Erkenntnissen, die wertvolle Einblicke in die digitale Nutzererfahrung ermöglichen.

Die Analyse der Homepage und der Kategorietaxonomie ist entscheidend für die erste Benutzererfahrung und beeinflusst, wie gut Benutzer die Webseite verstehen und nutzen können. Dabei werden sowohl die Struktur als auch das Design der Homepage untersucht, um sicherzustellen, dass die wichtigsten Inhalte leicht zugänglich sind und Benutzer effizient durch die Seite navigieren können. Zu den untersuchten Aspekten gehören der effektive Einsatz von Karussells, die Personalisierung von Inhalten und die Hervorhebung von Werbeaktionen. Zusätzlich wird die Kategorietaxonomie betrachtet, um potenzielle Überkategorisierung zu identifizieren und die Klarheit der Informationsarchitektur zu bewerten. Auch die Namensgebung der Kategorien wird geprüft, um sicherzustellen, dass sie für Nutzer verständlich und intuitiv sind. Die Hauptnavigation, einschließlich der Gestaltung von Mega-Drop-Down-Menüs und der visuellen Hierarchie der Menüpunkte, wird analysiert, um sicherzustellen, dass der Nutzerfluss gut strukturiert ist. Zwischenkategorie-Seiten, die inspirierende Inhalte oder kuratierte Produktlisten bereitstellen, werden ebenfalls einbezogen. Abschließend wird das gesamte Seitenlayout bewertet, um sicherzustellen, dass wichtige Elemente wie Fußzeilen, Newsletter-Dialoge und Rückgabebestimmungen sichtbar und zugänglich sind.

Die Sucherfahrung ist ein wesentlicher Bestandteil der Benutzerfreundlichkeit, insbesondere bei E-Commerce-Websites, da viele Benutzer die Suchfunktion nutzen, um gezielt nach Produkten zu suchen. Im Audit wird analysiert, wie gut die Suchmaschine verschiedene Arten von Suchanfragen unterstützt, zum Beispiel exakte Übereinstimmungen oder thematische Anfragen, die nicht spezifisch sind. Auch die Fähigkeit der Suchmaschine, auf unterschiedliche Schreibweisen oder kleinere Fehler zu reagieren, wird bewertet. Zusätzlich wird das Design und die Funktion des Suchfelds untersucht. Dies umfasst die Frage, ob Benutzer Suchbereiche manuell auswählen können oder ob die Auswahl automatisch erfolgt. Die Handhabung von Sonderzeichen und die Reaktionsfähigkeit des Suchfelds werden ebenfalls geprüft. Die Autocomplete-Funktion wird daraufhin bewertet, ob sie nützliche Vorschläge bietet, die den Benutzer bei der Suche unterstützen. Auch die Benutzerführung auf der Suchergebnisseite spielt eine wichtige Rolle. Die Darstellung der Ergebnisse sowie die Verfügbarkeit von Filter- und Sortieroptionen werden bewertet. Zudem wird darauf geachtet, dass alternative Suchvorschläge bereitgestellt werden, wenn keine Ergebnisse gefunden werden, um die Benutzererfahrung zu verbessern.

Die Gestaltung von Produktlisten und die Filtermechanismen sind entscheidend für die Benutzerfreundlichkeit einer Webseite, insbesondere bei großen E-Commerce-Websites mit umfangreichen Produktkatalogen. Das Audit untersucht das Listenlayout, um zu verstehen, ob eine Raster- oder Listenansicht für die jeweiligen Produkttypen besser geeignet ist. Das Schema für das Laden neuer Produkte, wie zum Beispiel "Endlos-Scrollen" oder "Seiten laden", wird ebenfalls analysiert, da dies den Zugang zu den Produkten beeinflusst. Auch die Darstellung von Produktvariationen innerhalb der Listenartikel wird untersucht, damit alle relevanten Informationen für Benutzer leicht zugänglich sind. Ein weiteres wichtiges Thema ist die Filter- und Sortierlogik. Die verfügbaren Filter sollen den Benutzerbedürfnissen entsprechen und es ermöglichen, Produkte einfach zu finden. Die Anzahl und Spezifität der Filteroptionen sowie die Klarheit und Sinnhaftigkeit der Sortieroptionen werden geprüft, damit Benutzer schnell die relevanten Produkte anzeigen können.

Produktdetailseiten sind ein entscheidender Punkt im Kaufprozess und haben großen Einfluss auf die Conversion-Rate. Die Analyse der Produktdetailseiten umfasst verschiedene Aspekte, wie das Layout der Seiten und die Nutzung horizontaler Registerkarten zur besseren Strukturierung der Informationen. Dadurch soll sichergestellt werden, dass die Benutzer die benötigten Informationen schnell finden können, ohne von zu vielen Details überfordert zu werden. Auch die Darstellung von wesentlichen Informationen, wie Preise, Produktbeschreibungen und Verfügbarkeit, wird untersucht, um sicherzustellen, dass diese leicht auffindbar sind. Die Produktbilder werden ebenfalls bewertet, um sicherzustellen, dass sie eine benutzerfreundliche Erfahrung bieten, zum Beispiel durch Zoom-Funktionen oder unterschiedliche Blickwinkel des Produkts. Der Kaufabschnitt, einschließlich der Platzierung und Gestaltung der "In den Warenkorb"-Schaltfläche, wird analysiert, da dies entscheidend für den Kaufprozess ist. Außerdem wird untersucht, wie Produktvariationen, wie Farben oder Größen, dargestellt werden, damit Benutzer diese leicht auswählen können.

Der Checkout-Prozess ist entscheidend, um Kaufabbrüche zu minimieren. Im Audit wird der Warenkorb und sein Verhalten analysiert, beispielsweise die Reaktion der Seite beim Hinzufügen eines Produkts zum Warenkorb. Es ist wichtig, dass Benutzer klar darüber informiert werden, wenn ein Produkt erfolgreich hinzugefügt wurde, um Unsicherheiten zu vermeiden. Auch die Gestaltung der Kunden- und Adressformulare wird bewertet, insbesondere hinsichtlich der Benutzerfreundlichkeit und Klarheit der Anweisungen. Die Handhabung von Datenschutzbelangen wird ebenfalls geprüft, damit Benutzer ihre Daten sicher eingeben können. Der Zahlungsfluss wird analysiert, um sicherzustellen, dass alle relevanten Zahlungsoptionen verfügbar und leicht zugänglich sind, einschließlich der Integration von Drittanbieter-Zahlungen wie PayPal. Zudem wird die Bestellbestätigungsseite darauf überprüft, dass sie alle wichtigen Informationen klar und vollständig darstellt, damit Benutzer keine offenen Fragen haben.

Die mobile Benutzererfahrung ist besonders wichtig, da das Nutzungsverhalten auf mobilen Geräten anders ist als auf Desktop-Webseiten. Daher wird die mobile Version der Seite mit der Desktop-Version verglichen, um Unterschiede in der Benutzerfreundlichkeit zu identifizieren. Die Größe der Hit-Bereiche für Touch-Interaktionen wird geprüft, da mobile Benutzer meist mit ihren Fingern navigieren, und es wichtig ist, dass Schaltflächen groß genug sind, um Fehlklicks zu vermeiden. Die mobile Navigation wird daraufhin untersucht, dass Benutzer problemlos zwischen den Seiten navigieren können. Auch die Platzierung von wichtigen Elementen, wie dem "In den Warenkorb"-Button, wird bewertet, um sicherzustellen, dass diese auf kleinen Bildschirmen gut sichtbar sind. Der mobile Checkout-Prozess wird ebenfalls analysiert, um sicherzustellen, dass der Kaufprozess für mobile Benutzer möglichst einfach und effizient ist. Dies umfasst unter anderem die Minimierung der Anzahl der erforderlichen Felder im Checkout.

Das Baymard Institut bietet die Möglichkeit, das Audit individuell an die spezifischen Anforderungen eines Unternehmens anzupassen. Zusätzliche Bereiche, wie die Kontoverwaltung und Rückgabeverfahren, können einbezogen werden, um alle Aspekte der Nutzererfahrung abzudecken. Für spezielle Branchen, wie zum Beispiel Reiseanbieter oder B2B-Unternehmen, können maßgeschneiderte Audits durchgeführt werden, die auf die jeweiligen Anforderungen zugeschnitten sind. Eine weitere Option ist die Wettbewerbsanalyse, bei der die UX-Performance vertraulich mit direkten Wettbewerbern verglichen wird, um gezielte Verbesserungspotenziale zu identifizieren. Diese Anpassungen ermöglichen eine präzisere Analyse, die auf die spezifischen Herausforderungen des jeweiligen Unternehmens eingeht.

Die UX-Audit-Dienstleistungen des Baymard Instituts können für unterschiedliche Zwecke eingesetzt werden, unter anderem für Benchmarking. Benchmarking ermöglicht es, die eigene UX-Leistung im Vergleich zu führenden Wettbewerbern und modernen Best-Practice-Webseiten zu bewerten. Dies hilft dabei, eine objektive Perspektive zu gewinnen und Verbesserungspotenziale zu identifizieren. Zudem können die Audits zur Verifizierung von Redesigns verwendet werden, bevor in die Entwicklung des finalen Codes investiert wird, um sicherzustellen, dass die neuen Designs tatsächlich zu einer verbesserten Nutzererfahrung führen. Die Audits eignen sich auch zur Überprüfung bereits optimierter Seiten, um letzte Feinheiten zu identifizieren, die eventuell noch optimiert werden können. Darüber hinaus können Unternehmen mithilfe der UX-Audits ihre Nutzererfahrung langfristig verfolgen und dokumentieren, was besonders hilfreich ist, um Fortschritte für Stakeholder sichtbar zu machen und sicherzustellen, dass die Optimierungen nachhaltig und effektiv sind.


\subsubsection{Forschungsmethoden und Richtlinien}
\label{forschungsmethoden_und_richtlinien}
Das Baymard Institut (Research Methodology – Baymard Institute, o. D.) verfolgt einen methodisch fundierten Ansatz zur Verbesserung der User Experience (UX) auf E-Commerce-Websites. Die Methodologie des Instituts basiert auf vier primären Forschungstechniken, welche zusammen ein umfassendes und detailliertes Richtlinienwerk zur Optimierung der Benutzererfahrung generieren. Die erste und eine der zentralen Methoden des Baymard-Institute ist der Einsatz von Usability-Tests nach dem "Think Aloud"-Protokoll. Dabei äußern die Nutzer während der Bearbeitung von Aufgaben laut ihre Gedanken und Überlegungen. Diese qualitative Methode erlaubt es den Forschenden, tiefgreifende Einsichten in die tatsächliche Interaktion der Nutzer mit einer Website zu gewinnen und potenzielle Usability-Probleme zu identifizieren, die andernfalls verborgen bleiben könnten. Im Rahmen dieser Tests wurden in 25 Runden mehr als 4.400 Sitzungen durchgeführt. Dabei wurden die Teilnehmer mit typischen Szenarien auf E-Commerce-Websites konfrontiert, beispielsweise dem Finden und Vergleichen von Produkten oder dem Abschließen eines Kaufvorgangs. Die identifizierten Probleme waren vielfältig und reichten von Schwierigkeiten bei der Navigation bis hin zu Missverständnissen in Bezug auf Produktinformationen oder technische Barrieren. Im Rahmen dieser Tests konnten insgesamt mehr als 34.000 Usability-Probleme identifiziert werden, welche anschließend in 707 UX-Richtlinien destilliert wurden. Die erarbeiteten Richtlinien bilden die Grundlage für die Konzeption einer konsistenten und qualitativ hochwertigen Benutzererfahrung, welche auf den tatsächlichen Bedürfnissen und Verhaltensmustern der Nutzer basiert.

In der vorliegenden Untersuchung wird zudem die Methode des manuellen Benchmarkings herangezogen, wie sie vom Baymard Institute in der Forschung Anwendung findet. Diese Technik zielt darauf ab, die tatsächliche Umsetzung der UX-Richtlinien auf bestehenden E-Commerce-Websites zu evaluieren und zu vergleichen. Im Rahmen der Studie wurden in 54 Runden von Benchmarking-Analysen insgesamt 251 der umsatzstärksten E-Commerce-Websites in den USA und Europa untersucht. Im Rahmen dessen wurde jede Website anhand der 707 UX-Richtlinien beurteilt, welche aus den zuvor durchgeführten Usability-Tests abgeleitet wurden. Die Bewertung erfolgte auf einer Skala von 1 bis 7, wobei die Webseiten durch UX-Experten anhand heuristischer Prinzipien geprüft wurden. 
Das Resultat dieser detaillierten Analyse ist eine umfassende Datenbank, welche über 275.000 UX-Performancebewertungen umfasst. Diese bieten einen tiefgreifenden Einblick in die Stärken und Schwächen der untersuchten Websites. Gleichzeitig wurden mehr als 175.000 Best-Practice-Beispiele zusammengetragen, welche aufzeigen, wie erfolgreiche E-Commerce-Websites spezifische UX-Herausforderungen bewältigt haben. Die identifizierten Best Practices stellen für Unternehmen, die eine Optimierung ihrer eigenen Webseiten anstreben, eine wertvolle Referenz dar, von der sie lernen können, wie sie die Stärken der Marktführer für sich nutzen können.

Die qualitativen Methoden werden durch Eye-Tracking-Tests ergänzt, welche einen tiefen Einblick in das visuelle Verhalten der Nutzer geben. Die Beobachtung der Blickbewegungen der Teilnehmer erfolgte mittels eines Tobii-Eye-Trackers. Dabei wurde analysiert, wie sich die Blicke über die verschiedenen Produktseiten verteilen. Die Analyse der Blickbewegungen erlaubt eine präzise Rekonstruktion derjenigen Elemente einer Seite, die die größte Aufmerksamkeit auf sich ziehen, sowie derjenigen, die möglicherweise übersehen werden. In einer separaten Studie wurden 32 Teilnehmer gebeten, auf verschiedenen E-Commerce-Websites nach Produkten zu suchen und diese zu erwerben. Die Eye-Tracking-Studie ermöglichte wertvolle Einblicke in die Wahrnehmung von Produktbildern, Beschreibungen und weiteren visuellen Inhalten durch Nutzer sowie deren Einfluss auf die Kaufentscheidung. Die genannten Tests erweisen sich insbesondere als nützlich zur Identifikation visueller Schwachstellen auf einer Website. Als Beispiele können hier schlecht platzierte Call-to-Action-Buttons oder ineffektiv gestaltete Produktbilder genannt werden. Die Ergebnisse liefern den Designern klare Hinweise bezüglich einer Optimierung der visuellen Hierarchie der Seite, sodass die Aufmerksamkeit der Nutzer auf die wichtigsten Elemente gelenkt wird.

Neben den qualitativen Ansätzen führt das Baymard Institute auch quantitative Studien durch, um auf Basis statistischer Verfahren Erkenntnisse über das Nutzerverhalten zu gewinnen. Im Rahmen von zwölf verschiedenen Studien wurden insgesamt 20.240 Teilnehmer befragt, um spezifische Aspekte der Benutzererfahrung zu beleuchten. Die Studien decken ein breites Spektrum an Themen ab, darunter die Gründe für Kaufabbrüche, das Vertrauen der Nutzer in SSL- und Sicherheitssiegel sowie die Häufigkeit von Fehlern bei CAPTCHA-Verifikationen. Der Vorteil quantitativer Daten besteht in der Möglichkeit, eine breitere Perspektive auf das Nutzerverhalten und die Präferenzen zu gewinnen sowie allgemeine Trends und Muster zu erkennen. Die Kombination der quantitativen Daten mit den qualitativen Erkenntnissen aus den Usability- und Eye-Tracking-Tests erlaubt die Erstellung eines umfassenden Bildes derjenigen Faktoren, welche die Benutzererfahrung in den unterschiedlichen Phasen des Kaufprozesses positiv oder negativ beeinflussen. Die Studien des Baymard-Institute liefern detaillierte Informationen über die Funktionen von Nutzerkonten, die von den Nutzer am häufigsten verwendet werden. Ebenso werden die Darstellungsweisen von kostenlosem Versand untersucht, um deren Einfluss auf die Konversion zu bestimmen.

Die aus den verschiedenen Methoden gewonnenen Erkenntnisse werden in ein umfassendes Bewertungssystem integriert, welches jede der 707 UX-Richtlinien anhand zweier Hauptkriterien klassifiziert: der Schwere des Problems und der Häufigkeit seines Auftretens. Die Beurteilung der Schwere eines Problems erfolgt anhand des Grads der Beeinträchtigung der Nutzererfahrung. Die Kategorisierung der Probleme erfolgt anhand eines dreistufigen Schemas, welches zwischen temporären, aktiven und gravierenden Störungen unterscheidet. Temporäre Störungen bezeichnen Beeinträchtigungen, die lediglich zu einer kurzzeitigen Verringerung der Nutzerzufriedenheit führen. Aktive Störungen resultieren in einer aktiven Suche des Nutzers nach einer Lösung, während gravierende Störungen ein Verlassen der Website ohne Abschluss der ursprünglichen Aufgabe zur Folge haben. Die Häufigkeit bezieht sich auf die potenzielle Auftrittswahrscheinlichkeit des Problems bei den Nutzern. Diese kann sich auf eine geringe Anzahl von Nutzern oder auf alle Nutzer beziehen. Die Kombination beider Faktoren resultiert in einer Wichtigkeitsbewertung, welche die Basis für die Priorisierung von Optimierungsvorschlägen darstellt. Des Weiteren werden einige Richtlinien als "verpasste Chancen" klassifiziert, wenn sie von einer signifikanten Anzahl an Websites inkorrekt implementiert werden, obwohl sie ein beträchtliches Potenzial zur Beeinflussung der User Experience aufweisen könnten. Andere Richtlinien, deren korrekte Umsetzung durch 80 \% oder mehr der Websites bestätigt wurde, werden als "Web-Konventionen" bezeichnet. Sie stellen somit Best Practices dar, die als allgemeingültig betrachtet werden können. Zuletzt gibt es noch die Kategorie "Low-Cost". Hier werden Veränderungen an der Website genannt die ohne große Kosten oder Aufwand einen positiven Effekt auf die Nutzererfahrungen haben. 

Das vom Baymard Institute entwickelte Benchmarking-System bewertet die Umsetzung der 707 UX-Richtlinien auf den untersuchten Websites. Die Punktzahl, welche jede Website erhält, basiert auf dem ersten Eindruck, den ein Nutzer von der Website gewinnt. Die ermittelte Punktzahl wird durch einen dynamischen Normalisierungsfaktor angepasst, um eine konsistente Bewertung über die Zeit zu gewährleisten und eine Anpassung an eine etwaige Weiterentwicklung der allgemeinen UX-Standards zu ermöglichen. Folglich ist es Unternehmen möglich, ihre Leistung im Kontext der weltweit führenden E-Commerce-Websites zu vergleichen und auf Basis dessen gezielte Optimierungen vorzunehmen.

