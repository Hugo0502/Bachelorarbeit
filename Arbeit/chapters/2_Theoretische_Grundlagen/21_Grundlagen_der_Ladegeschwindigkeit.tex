\section{Definitionen und Ansätze}
\label{sec:definitionen_und_ansatze}
Die Ladegeschwindigkeit einer Website, auch als Seitenladezeit oder Seitengeschwindigkeit bezeichnet, ist ein zentraler Aspekt der Performance einer Webseite und spielt eine entscheidende Rolle für die Nutzererfahrung sowie die Suchmaschinenoptimierung (SEO) (Wilkinson, 2024). Verschiedene Definitionen betonen dabei unterschiedliche Nuancen des Begriffs.

\subsection{Definition der Ladegeschwindigkeit}
\label{sec:definition_der_ladegeschwindigkeit}
Elbeyoglu (2021) beschreibt die Seitenladezeit als die Zeitspanne, die erforderlich ist, damit alle Informationen auf einer Webseite vollständig angezeigt werden. Diese Ladezeit wird auch als Seitengeschwindigkeit bezeichnet und misst die Effizienz des Ladevorgangs. Der Fokus liegt hier auf der Vollständigkeit der Darstellung der Seite, was eine benutzerorientierte Perspektive betont.

Die Definition von Page Load Time (2023) beschreibt die Seitenladezeit als die Dauer, die eine Webseite benötigt, um vollständig zu laden, gemessen in Sekunden. Zwar wird hier die vollständige Ladung als zentraler Punkt genannt, jedoch wird auch die quantitative Messbarkeit der Ladezeit betont, wodurch die Ladezeit als wichtiger Indikator für die Leistungsfähigkeit einer Webseite hervorgehoben wird.

Wilkinson (2024) legt den Schwerpunkt auf die Schnelligkeit des Ladevorgangs. Nach dieser Definition misst die Seitengeschwindigkeit, wie schnell der Inhalt einer Seite geladen wird, wobei der Begriff Ladegeschwindigkeit synonym verwendet wird. Dieser Ansatz konzentriert sich stärker auf die Effizienz des Ladevorgangs, ohne die vollständige Anzeige des gesamten Inhalts als zwingenden Maßstab zu betrachten.

Die Quelle „What Is Website Loading Speed?“ (o. D.) bietet eine weitere Betrachtung, indem sie die Ladegeschwindigkeit als die Zeit beschreibt, die benötigt wird, um den gesamten Inhalt einer Website auf dem Gerät des Nutzers darzustellen, nachdem diese durch Eingabe einer URL, das Anklicken eines Links oder eine Weiterleitung aufgerufen wurde. Diese Definition integriert explizit die Nutzerperspektive, indem der gesamte Prozess von der Benutzeraktion bis zur vollständigen Anzeige der Inhalte auf dem Endgerät des Nutzers beschrieben wird.

\subsection{Der Ladeprozess als mehrstufige Einheit}
\label{sec:der_ladeprozess_als_mehrstufige_einheit}
Web.dev, eine von Google gestützte Seite, bietet einen weiteren Ansatz zur Ladegeschwindigkeit. Anstey und Pavic (2019) betrachten das Laden einer Webseite nicht als einen einzelnen Moment, sondern als einen mehrstufigen Prozess. Dieser Ansatz hebt hervor, dass einzelne Metriken, wie etwa die DOMContentLoaded-Zeit, kein vollständiges Bild des Nutzererlebnisses vermitteln können. Stattdessen wird die Ladegeschwindigkeit in drei Etappen mit vier Messwerten unterteilt.

Die erste Etappe lautet "Is it happening?" und überprüft, ob die Navigation erfolgreich begonnen hat und ob der Server reagiert. In dieser Etappe werden die Metriken "First Paint" und "First Contentful Paint" gemessen, die die Zeit beschreiben, bis etwas Sichtbares, sei es ein Bild oder Text, auf der Seite erscheint.

Die zweite Etappe heißt "Is it useful?" und markiert den Zeitpunkt, an dem genügend Inhalt geladen ist, sodass sich der Nutzer sinnvoll mit der Seite beschäftigen kann. Dies ist der Moment, in dem erkennbar wird, ob die Website für den Nutzer von Interesse ist. Diese Phase wird durch die Metrik "First Meaningful Paint" beschrieben.

Die dritte Etappe heißt "Is it usable?" und beschreibt den Zeitpunkt, ab dem der Nutzer tatsächlich mit der Webseite interagieren kann. Hierbei kommt die Metrik "Time to Interactive" zum Einsatz, die misst, wann die Webseite  funktionsfähig ist und  interaktiven Elemente genutzt werden können.

Nach diesen drei Etappen kann der initiale Ladeprozess als abgeschlossen betrachtet werden. Optional kann jedoch eine vierte Etappe hinzugefügt werden: "Is it delightful?". Diese beschreibt die Qualität der Nutzung der Webseite, wobei Faktoren wie reibungslose Nutzung, das Fehlen von Verzögerungen und eine flüssige Bedienung im Vordergrund stehen. Auch hier werden weitere spezifische Messwerte benötigt, um die Benutzerfreundlichkeit und die natürliche Interaktion zu bewerten.

Insgesamt wird aus diesen verschiedenen Definitionen deutlich, dass die Ladegeschwindigkeit sowohl technische Aspekte der Performance als auch die Benutzererfahrung umfasst. Sie beeinflusst direkt die Zufriedenheit der Nutzer und kann sich direkt auf das Engagement und die Wiederkehrwahrscheinlichkeit auswirken.




\section{Einflussfaktoren auf die Ladegeschwindigkeit und deren Lösungen}
\label{sec:einflussfaktoren_auf_die_ladegeschwindigkeit_und_deren_lösungen}


